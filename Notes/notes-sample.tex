% !TEX program = pdflatex
\documentclass{tufte-handout}

\title{\centering Course:  Course Name}
\author{I'm the author}


\date{\today} % without \date command, current date is supplied

%\geometry{showframe} % display margins for debugging page layout

\usepackage{graphicx} % allow embedded images
  \setkeys{Gin}{width=\linewidth,totalheight=\textheight,keepaspectratio}
\usepackage{amsmath}  % extended mathematics
\usepackage{booktabs} % book-quality tables
\usepackage{units}    % non-stacked fractions and better unit spacing
\usepackage{multicol} % multiple column layout facilities
\usepackage{lipsum}   % filler text
\usepackage{fancyvrb} % extended verbatim environments
  \fvset{fontsize=\normalsize}% default font size for fancy-verbatim environments

% Standardize command font styles and environments
\newcommand{\doccmd}[1]{\texttt{\textbackslash#1}}% command name -- adds backslash automatically
\newcommand{\docopt}[1]{\ensuremath{\langle}\textrm{\textit{#1}}\ensuremath{\rangle}}% optional command argument
\newcommand{\docarg}[1]{\textrm{\textit{#1}}}% (required) command argument
\newcommand{\docenv}[1]{\textsf{#1}}% environment name
\newcommand{\docpkg}[1]{\texttt{#1}}% package name
\newcommand{\doccls}[1]{\texttt{#1}}% document class name
\newcommand{\docclsopt}[1]{\texttt{#1}}% document class option name
\newenvironment{docspec}{\begin{quote}\noindent}{\end{quote}}% command specification environment
%%%%%%%%%%%%%%%%%%%%%%%%%%%%%%%%%%%%%%%%%%%%%%%%%%%%%%%%%%%%%%%%%%%%%%%%%%%%%%%%%%%%%%%%%%%%%%%%%%%%%%
% add numbers to chapters, sections, subsections
\setcounter{secnumdepth}{2}
\usepackage{xcolor}
\definecolor{g1}{HTML}{077358}
\definecolor{g2}{HTML}{00b096}
% chapter format  %(if you use tufte-book class)
%\titleformat{\chapter}%
%{\huge\rmfamily\itshape\color{red}}% format applied to label+text
%{\llap{\colorbox{red}{\parbox{1.5cm}{\hfill\itshape\huge\color{white}\thechapter}}}}% label
%{2pt}% horizontal separation between label and title body
%{}% before the title body
%[]% after the title body

% section format
\titleformat{\section}%
{\normalfont\Large\itshape\color{g1}}% format applied to label+text
{\llap{\colorbox{g1}{\parbox{1.5cm}{\hfill\color{white}\thesection}}}}% label
{1em}% horizontal separation between label and title body
{}% before the title body
[]% after the title body

% subsection format
\titleformat{\subsection}%
{\normalfont\large\itshape\color{g2}}% format applied to label+text
{\llap{\colorbox{g2}{\parbox{1.5cm}{\hfill\color{white}\thesubsection}}}}% label
{1em}% horizontal separation between label and title body
{}% before the title body
[]% after the title body

%%%%%%%%%%%%%%%%%%%%%%%%%%%%%%%%%%%%%%%%%%%%%%%%%%%%%%%%%%%%%%%%%%%%%%%%%%%%%%%%%%%%%%%%%%%%%%%%%%%%%%
\usepackage{color-tufte}
%%%%%%%%%%%%%%%%%%%%%%%%%%%%%%%%%%%%%%%%%%%%%%%%%%%%%%%%%%%%%%%%%%%%%%%%%%%%%%%%%%%%%%%%%%%%%%%%


\begin{document}

\maketitle% this prints the handout title, author, and date

\begin{abstract}
\noindent
A simple notes template. Inspired by Tufte-\LaTeX class and beautiful notes by \begin{verbatim*}
	https://github.com/abrandenberger/course-notes
\end{verbatim*}
\end{abstract}

%\printclassoptions


\section{Page Layout}\label{sec:page-layout}

\lipsum[1][1-8]\footnote[1]{Footnotes will appear on the margins}

\begin{definition}%%  [can be kept empty]
	Here's is the beautiful Schr\"odinger equation
	\[ i\hbar {\frac {\partial }{\partial t}}\Psi (x,t)=
	\left[-{\frac {\hbar ^{2}}{2m}}{\frac {\partial ^{2}}{\partial x^{2}}}+V(x,t)\right]\Psi (x,t)\]
\end{definition}

\subsection{Headings}\label{sec:headings}


\marginnote{\begin{proof}[Proof (Theorem 1.1)] 
		
		\lipsum[1][1-3]\end{proof}}
\begin{theorem}%%  [can be kept empty]
	\lipsum[1][1-3] %% for dummy text
\end{theorem}

\begin{lemma}%%  [can be kept empty]
	\lipsum[1][1-3] %% for dummy text
	
\end{lemma}
\begin{proof}
	\lipsum[1][1-5]
\end{proof}

%\marginnote{\begin{proof}\lipsum[1][1-3]\end{proof}}

\begin{corollary}%%  [can be kept empty]
	\lipsum[1][1-3] %% for dummy text
\end{corollary}

\begin{proposition}
	\lipsum[1][1-3] %% for dummy text
\end{proposition}
\begin{problem}
	\lipsum[1][1-2]
\end{problem}

\begin{proof}
	\lipsum*[1]
\end{proof}


\end{document}
